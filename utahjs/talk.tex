\documentclass[xcolor=svgnames,compress]{beamer}

\newcommand{\blue}[1]{\textcolor{DarkBlue}{#1}}
\newcommand{\white}[1]{\textcolor{White}{#1}}
\newcommand{\gray}[1]{\textcolor{black!70}{#1}}
\newcommand{\red}[1]{\textcolor{FireBrick}{#1}}
\newcommand{\code}[1]{\textcolor{blue}{\texttt{#1}}}

\newcommand{\separator}[1]{
  {
    \setbeamercolor{background canvas}{bg=LightSteelBlue}
    \frame[plain]{
      \begin{center}
        \gray{\Huge \bf #1}
      \end{center}
    }
  }
}

\newcommand{\divider}[1]{
  {
    \setbeamercolor{background canvas}{bg=Gold}
    \frame[plain]{
      \begin{center}
        \gray{\Huge \bf #1}
      \end{center}
    }
  }
}


\newcommand{\fig}[2]{\centerline{\includegraphics[height={#1}]{figures/{#2}}}}
\newcommand{\figw}[2]{\centerline{\includegraphics[width={#1}]{figures/{#2}}}}
\newcommand{\graph}[2]{\centerline{\includegraphics[height={#1}]{graphs/{#2}}}}
\newcommand{\graphw}[2]{\centerline{\includegraphics[width={#1}]{graphs/{#2}}}}
\newcommand{\headline}[1]{\centerline{\small \blue{#1}}}

\usetheme{Frankfurt}

\setbeamertemplate{navigation symbols}{}
\setbeamertemplate{mini frames}{}
\setbeamercolor{section in head/foot}{bg=black!70}
\setbeamertemplate{title page}[default][colsep=-4bp,rounded=true]
\setbeamercolor{title}{fg=white,bg=black!70,size=\huge,series=\bfseries}
\setbeamercolor{frametitle}{fg=black!70,bg=white}
\setbeamerfont{frametitle}{series=\bfseries}
\setbeamertemplate{items}[circle]
\setbeamercolor{itemize item}{fg=black!40}
\setbeamercolor{button}{bg=black!70,fg=white}
\setbeamertemplate{frametitle}[default][colsep=-4bp,rounded=false,shadow=false]

\setbeamertemplate{headline}
{%
  \begin{beamercolorbox}[ht=3ex,dp=1.125ex,%
      leftskip=.3cm,rightskip=.3cm plus1fil]{section in head/foot}
    \usebeamerfont{section in head/foot}\usebeamercolor[fg]{section in head/foot}%
    \insertsectionnavigationhorizontal{0.95\textwidth}{}{}
  \end{beamercolorbox}%
}

\setbeamertemplate{footline}{\hfill{\hspace*{2em}\insertframenumber/\inserttotalframenumber\hspace*{2ex}\vskip0pt}}

% PDF settings
\hypersetup{%
  pdftitle={Civic Hacking},%
  pdfauthor={Daniel Zappala},%
  pdfsubject={},%
  pdfkeywords={}%
}

\title{Civic Hacking}
\author{Daniel Zappala}
\institute{Computer Science\\Brigham Young University}
\date{\fig{1cm}{logo.png}\\\href{https://osl.byu.edu/}{\beamergotobutton{Lab Home}} \href{https://github.com/byu-osl/}{\beamergotobutton{Github}}}


\begin{document}

\frame[plain]{\titlepage}

\section{}

{
\setbeamercolor{background canvas}{bg=LightSteelBlue}
\frame[plain]{
  \vspace{0.25cm}
  {\large \bf Civic Hacking}
  \vspace{1cm}\\
  building apps to help local government connect with citizens\\
  \hspace{5cm}{\em -- BYU Open Source Lab}\\
  \vspace{1cm}
  helping cities create 21st century interfaces to government that are simple, beautiful and easy to use\\
  \hspace{5cm}{\em -- Code for America}\\
  \vspace{1cm}
  \href{http://embed.ted.com/talks/jennifer_pahlka_coding_a_better_government.html}{\beamergotobutton{Jennifer Pahlka TED Talk}}
}
}

\divider{quick\\\vspace{0.75cm}cheap\\\vspace{0.75cm}simple\\\vspace{0.75cm}beautiful\\\vspace{1cm}open}

\frame[plain]{
  \figw{13cm}{blightstatus}
}

\frame[plain]{
  \figw{13cm}{honolulu-answers}
}

\frame[plain]{
  \figw{13cm}{preparedly}
}

\frame[plain]{
  \figw{13cm}{crime-in-chicago}
}

\frame[plain]{
  \figw{13cm}{textizen}
}

\frame[plain]{
  \figw{13cm}{schoolbus}
}

\frame{
  \begin{itemize}
  \item really vibrant space
  \item lots of excitement about open government
  \item big on open source {\em and} startups
  \item opportunities
    \begin{itemize}
    \item \href{http://opencityapps.org}{\beamergotobutton{Open City Apps}}
    \item \href{http://codeforamerica.org}{\beamergotobutton{Code for America}}
    \item \href{http://sunlightfoundation.com/about/grants/opengovgrants/}{\beamergotobutton{Sunlight Foundation OpenGov Grants}}
    \item \href{http://www.knightfoundation.org/funding-initiatives/tech-engagement/}{\beamergotobutton{Knight Foundation Tech for Engagement}}
    \end{itemize}
  \end{itemize}
}

\separator{BYU Open Source Lab}

\frame[plain]{
  \figw{13cm}{citizen-budget}
  \href{http://citizenbudget.org}{\beamergotobutton{citizenbudget.org}}
}

\frame{
  \frametitle{Citizen Budget}
  \begin{itemize}
  \item started with a Code for America project, had students
    customize it, eventually rewrote it
  \item deployed at \href{http://budget.cedarhills.org}{\beamergotobutton{budget.cedarhills.org}}
  \item plans to improve it
    \begin{itemize}
    \item write a Javascript front end with Angular or Ember
    \item enable staff to edit explanations on each page
    \item allow user comments and responses from staff
    \item write additional converters for municipal budget packages
    \item possibly convert into SaaS, charge cities for storage
    \end{itemize}
  \end{itemize}
}

\frame{
  \frametitle{The Next App}
  \begin{itemize}
  \item customer service portal
    \begin{itemize}
    \item resident takes a photo, enters complaint in Smart Phone app
    \item begins a dialogue with city staff as they work on fixing the problem
    \item problems marked Open, Working, Closed
    \end{itemize}
  \item staff uses web app and phones to track progress of customer service
    items
  \item report track record to mayor on a weekly/monthly basis
  \item \href{http://www.cityofboston.gov/doit/apps/citizensconnect.asp}{\beamergotobutton{Citizens Connect (Boston)}}
  \item \href{http://servicetracker.cityofchicago.org/}{\beamergotobutton{ServiceTracker for City of Chicago}}
  \item \href{http://open311.org/}{\beamergotobutton{Open311.org}}
  \item \href{https://github.com/codeforamerica/srtracker}{\beamergotobutton{GitHub}}
  \end{itemize}
}

\frame{
  \frametitle{The Next App}
  \begin{itemize}
  \item digital feedback
    \begin{itemize}
    \item city posts short, targeted questions on current agenda items
    \item {\em what should the curfew be for city parks?}
    \item residents can provide feedback, vote each other up or down
    \item similar to Textizen, but via smartphone app, more interactive
    \end{itemize}
  \item council can use feedback to guide decisions
  \end{itemize}
}

\frame{
  \frametitle{Framework}
  \begin{itemize}
    \item design an open source framework for city apps
    \begin{itemize}
    \item hosted in one place instead of tens of different apps
    \item plugin architecture so cities can activate different apps on
      demand
    \item hostable or SaaS
    \end{itemize}
  \item the Wordpress for civic hacking
  \end{itemize}
}

\divider{Vision}

\frame{
  \frametitle{Vision}
  \begin{itemize}
  \vfill \item BYU Open Source Lab hosts R\&D for city apps
      \begin{itemize}
      \item joint research with Political Science, Digital Arts,
        Computer Science to study citizen engagement through deployed apps
      \item funding via Sunlight Foundation, Knight Foundation, others
      \end{itemize}
    \vfill \item collaboration with developer community through Code for America
      Brigade
      \begin{itemize}
        \item organize a hack night that meets at least once a month
        \item foster a relationship with a local government partner
      \end{itemize}
    \vfill \item can stay open source, can get help to transition to a startup
      \begin{itemize}
      \item Textizen
      \item BlightStatus $\rightarrow$ Civic Insight
      \item Honolulu Answers $\rightarrow$ Department of Better Technology
      \end{itemize}
  \end{itemize}
}

\separator{Who Wants to Help Out?\\\vspace{1cm}daniel.zappala@gmail.com\\\vspace{1cm}zappala.byu.edu}

\end{document}
