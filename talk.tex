\documentclass[xcolor=svgnames,compress]{beamer}

\newcommand{\blue}[1]{\textcolor{DarkBlue}{#1}}
\newcommand{\white}[1]{\textcolor{White}{#1}}
\newcommand{\gray}[1]{\textcolor{black!70}{#1}}
\newcommand{\red}[1]{\textcolor{FireBrick}{#1}}
\newcommand{\code}[1]{\textcolor{blue}{\texttt{#1}}}

\newcommand{\separator}[1]{
  {
    \setbeamercolor{background canvas}{bg=LightSteelBlue}
    \frame[plain]{
      \begin{center}
        \gray{\Huge \bf #1}
      \end{center}
    }
  }
}

\newcommand{\divider}[1]{
  {
    \setbeamercolor{background canvas}{bg=Gold}
    \frame[plain]{
      \begin{center}
        \gray{\Huge \bf #1}
      \end{center}
    }
  }
}


\newcommand{\fig}[2]{\centerline{\includegraphics[height={#1}]{figures/{#2}}}}
\newcommand{\figw}[2]{\centerline{\includegraphics[width={#1}]{figures/{#2}}}}
\newcommand{\graph}[2]{\centerline{\includegraphics[height={#1}]{graphs/{#2}}}}
\newcommand{\graphw}[2]{\centerline{\includegraphics[width={#1}]{graphs/{#2}}}}
\newcommand{\headline}[1]{\centerline{\small \blue{#1}}}

\usetheme{Frankfurt}

\setbeamertemplate{navigation symbols}{}
\setbeamertemplate{mini frames}{}
\setbeamercolor{section in head/foot}{bg=black!70}
\setbeamertemplate{title page}[default][colsep=-4bp,rounded=true]
\setbeamercolor{title}{fg=white,bg=black!70,size=\huge,series=\bfseries}
\setbeamercolor{frametitle}{fg=black!70,bg=white}
\setbeamerfont{frametitle}{series=\bfseries}
\setbeamertemplate{items}[circle]
\setbeamercolor{itemize item}{fg=black!40}
\setbeamercolor{button}{bg=black!70,fg=white}
\setbeamertemplate{frametitle}[default][colsep=-4bp,rounded=false,shadow=false]

\setbeamertemplate{headline}
{%
  \begin{beamercolorbox}[ht=3ex,dp=1.125ex,%
      leftskip=.3cm,rightskip=.3cm plus1fil]{section in head/foot}
    \usebeamerfont{section in head/foot}\usebeamercolor[fg]{section in head/foot}%
    \insertsectionnavigationhorizontal{0.95\textwidth}{}{}
  \end{beamercolorbox}%
}

\setbeamertemplate{footline}{\hfill{\hspace*{2em}\insertframenumber/\inserttotalframenumber\hspace*{2ex}\vskip0pt}}

% PDF settings
\hypersetup{%
  pdftitle={The Open Government Movement},%
  pdfauthor={Daniel Zappala},%
  pdfsubject={},%
  pdfkeywords={}%
}

\title{The Open Government Movement\\{\small Revolutionizing Citizen Engagement With Transparency and Technology}}
\author{Daniel Zappala}
\institute{Brigham Young University\\City of Cedar Hills}
\date{\fig{1cm}{logo.png}}



\begin{document}

\frame[plain]{\titlepage}

\section{}

\frame[plain]{
  \frametitle{Open Government}
  \begin{enumerate}
  \item bring government into the digital age
  \item make public data open by default
  \item create new opportunities for civic engagement through the use of technology
  \end{enumerate}
  \hspace{5cm}{\em \footnotesize -- Code for America}
}

\separator{How to be Open}

\frame{
  \frametitle{Interact Through Social Media}
  \figw{9cm}{facebook-curfew}
}

\frame{
  \frametitle{Encourage City Council Blogs}
  \figw{12cm}{cedarhillsblog}
}


\frame[plain]{
  \begin{center}
    \gray{\Huge \bf Build an App}
  \end{center}
}

\frame[plain]{
  \figw{13cm}{blightstatus}
}

\frame[plain]{
  \figw{13cm}{honolulu-answers}
}

\frame[plain]{
  \figw{13cm}{preparedly}
}

\frame[plain]{
  \figw{13cm}{crime-in-chicago}
}

\frame[plain]{
  \figw{13cm}{textizen}
}

\frame[plain]{
  \figw{13cm}{schoolbus}
}


\divider{Wait a Minute!\\\vspace{1cm} How Does My City Get An App?}

{
\setbeamercolor{background canvas}{bg=LightSteelBlue}
\frame[plain]{
  \vspace{0.25cm}
  {\large \bf Civic Hacking}
  \vspace{1cm}\\
  building apps to help local government connect with citizens\\
  \hspace{5cm}{\em -- BYU Open Source Lab}\\
  \vspace{1cm}
  helping cities create 21st century interfaces to government that are simple, beautiful and easy to use\\
  \hspace{5cm}{\em -- Code for America}\\
  \vspace{1cm}
  \href{http://embed.ted.com/talks/jennifer_pahlka_coding_a_better_government.html}{\beamergotobutton{Jennifer Pahlka TED Talk}}
}
}

\frame{
  \frametitle{How To Get an App}
  \begin{itemize}
  \item commit to open government data
    \begin{itemize}
    \item housing violations
    \item street sweeping schedule
    \item budgets and finances
    \end{itemize}
  \item identify a small, well-defined need
  \item find interested volunteers
    \begin{itemize}
    \item talk to \href{http://www.codeforamerica.org/}{\beamergotobutton{Code for America}} about organizing a Brigade
    \end{itemize}
  \item encourage collaboration between city officials, and volunteer
    developers, designers and researchers
  \item often results in an app for a city and a startup opportunity
    for the developers
  \end{itemize}
}

\frame{
  \frametitle{Open Source}
  \begin{itemize}
  \item code is available, free of charge, for anyone to use or modify
    \begin{itemize}
    \item licensing terms vary
    \item typically allows commercial use
    \item may require to to contribute your changes so everyone benefits
    \end{itemize}
  \item why provide code for free?
    \begin{itemize}
    \item public service
    \item collaboration
    \item security
    \item reputation
    \item sell service and support
    \end{itemize}
  \item often requires some expertise to deploy it
  \item definitely requires expertise to modify it
  \end{itemize}
}
    
\separator{BYU Open Source Lab}

\frame[plain]{
  \figw{13cm}{citizen-budget}
  deployed at \href{http://budget.cedarhills.org}{\beamergotobutton{budget.cedarhills.org}}
}

\frame{
  \frametitle{Citizen Budget}
  \begin{itemize}
  \item started with a Code for America project, had students
    customize it, eventually rewrote it
  \item any city can use it, see  \href{http://citizenbudget.org}{\beamergotobutton{citizenbudget.org}}
  \item plans to improve it
    \begin{itemize}
    \item enable staff to edit explanations on each page
    \item allow user comments and responses from staff
    \item write additional converters for municipal budget packages
    \item convert into a centralized site, available for any city, small fee
      for maintenance and support
    \end{itemize}
  \end{itemize}
}

\frame{
  \frametitle{The Next App}
  \begin{itemize}
  \item customer service portal
    \begin{itemize}
    \item resident takes a photo, enters complaint in Smart Phone app
    \item begins a dialogue with city staff as they work on fixing the problem
    \item problems marked Open, Working, Closed
    \end{itemize}
  \item staff uses web app and phones to track progress of customer service
    items
  \item report track record to mayor on a weekly/monthly basis
  \end{itemize}
}

\frame{
  \frametitle{The Next App}
  \begin{itemize}
  \item digital feedback
    \begin{itemize}
    \item city posts short, targeted questions on current agenda items
    \item {\em what should the curfew be for city parks?}
    \item residents can provide feedback, vote each other up or down
    \item similar to Textizen, but via smartphone app, more interactive
    \end{itemize}
  \item council can use feedback to guide decisions
  \end{itemize}
}

\divider{Vision}

\frame{
  \frametitle{Vision}
  \begin{itemize}
  \vfill \item BYU Open Source Lab hosts R\&D for city apps
      \begin{itemize}
      \item joint research with Political Science, Digital Arts,
        Computer Science to study citizen engagement through deployed apps
      \item funding via Sunlight Foundation, Knight Foundation, others
      \end{itemize}
    \vfill \item collaboration with developer community through Code for America
      Brigade
      \begin{itemize}
        \item organize a hack night that meets at least once a month
        \item foster relationships with local government partners
      \end{itemize}
    \vfill \item foster service and entrepreneurship
      \begin{itemize}
      \item Textizen
      \item BlightStatus $\rightarrow$ Civic Insight
      \item Honolulu Answers $\rightarrow$ Department of Better Technology
      \end{itemize}
  \end{itemize}
}

\separator{Interested?\\\vspace{1cm}daniel.zappala@gmail.com\\\vspace{1cm}zappala.byu.edu}

\end{document}
